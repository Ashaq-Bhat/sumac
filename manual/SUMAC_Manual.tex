\documentclass[10pt]{report}
\usepackage{geometry} 
\usepackage{indentfirst}
\usepackage{hyperref}
\usepackage{color}
\usepackage{comment}
\usepackage[pdftex]{graphicx}  
\usepackage{caption}
\usepackage{natbib}
\usepackage{mathtools}
\usepackage{units}
\usepackage{booktabs}
\usepackage{authblk}
\renewcommand{\baselinestretch}{1.5}
\geometry{a4paper} 
\bibliographystyle{apalike}

%----------------------------------------
%AUTHORS
%----------------------------------------
\title{SUMAC: Supermatrix Constructor version 1.0 Manual}
\author{William A. Freyman\thanks{freyman@berkeley.edu}}
\affil{Department of Integrative Biology, University of California, Berkeley}

\date{}

%-----------------------------------------------------------------------------------------------------------------
% BEGIN DOCUMENT
%-----------------------------------------------------------------------------------------------------------------
\begin{document}
\maketitle

\tableofcontents

%----------------------------------------
% INTRODUCTION
%----------------------------------------

\chapter{Introduction}

\section{About SUMAC}

SUMAC (Supermatrix Constructor) is a Python package to 
data mine GenBank and construct
and evaluate phylogenetic supermatrices. 
It is designed to be run as a command-line program, though
the modules can also be imported and used in other Python scripts.
SUMAC will assemble
supermatrices for any taxonomic group recognized in GenBank,
and is optimized to run on multicore processors and clusters by utilizing multiple parallel processes.

When run from the command-line, SUMAC will perform a number of steps to create
the phylogenetic supermatrix. 
First, SUMAC will download the GenBank database for the specified GenBank division (PLN, MAM, etc).
SUMAC will then build clusters of homologous sequences in one of two ways:
(1) perform exhaustive all-by-all BLAST comparisons of each ingroup and outgroup sequence
and use a single-linkage hierarchical clustering algorithm, or
(2) BLAST each ingroup and outgroup sequence against user provided guide sequences
that define each cluster.
SUMAC then discards clusters that are not phylogenetically informative ($< 4$ taxa), 
and then aligns each cluster of sequences using MAFFT.
Finally, the alignments are concatenated by species name (using the GenBank taxonomy) 
creating a supermatrix. A number of metrics are then calculated on the supermatrix, 
a graph indicating taxon coverage density is generated, and spreadsheets (in CSV format)
are produced with information about each DNA region and GenBank accession used in 
the supermatrix. 
There are many options described in
detail later in this manual. 


\section{Installation}

\subsection{Dependencies}

The following dependencies must be installed to run SUMAC:

\begin{itemize}
\item \texttt{Python 2.7}
\item \texttt{Biopython}
\item \texttt{matplotlib}
\item \texttt{MAFFT v6.9+}
\item \texttt{BLAST+}
\end{itemize}

\subsection{Installing Dependencies on Linux}

The following commands install the requirements for Debian GNU/Ubuntu Linux systems:

\begin{verbatim}
sudo pip install numpy
sudo pip install matplotlib
sudo pip install biopython
sudo apt-get install ncbi-blast+
sudo apt-get install mafft
\end{verbatim}
%git clone https://github.com/biopython/biopython.git
%cd biopython
%python setup.py build
%python setup.py test
%sudo python setup.py install

TODO: test installation!!

\subsection{Installing Requirements on Mac}

TODO!

\subsection{Installing SUMAC}

TODO: test pip to install...

\begin{verbatim}
python setup.py install
\end{verbatim}

\section{License and Warranty}
SUMAC is free software; you can redistribute it and/or modify it under the terms of the GNU General Public License as published by the Free Software Foundation; either version 3 of the License, or (at your option) any later version.

The program is distributed in the hope that it will be useful, but WITHOUT ANY WARRANTY; without even the implied warranty of MERCHANTABILITY or FITNESS FOR A PARTICULAR PURPOSE. See the GNU General Public License for more details \\ (http://www.gnu.org/copyleft/gpl.html).

%----------------------------------------
% Quick Start Tutorial
%----------------------------------------

\chapter{Quick Start Tutorial}

This chapter provides the quick way to get started using SUMAC. 
There are many details described in Chapter 3 that
would likely be helpful.

\section{Construct a Supermatrix}

The most basic usuage of SUMAC is to build a supermatrix with the following
command:

\begin{verbatim}
python -m sumac -d pln -i Onagraceae -o Lythraceae
\end{verbatim}

This command is an example of the minimum amount of input required
to run SUMAC. The first part of the command \texttt{python -m sumac}
runs the SUMAC module. The \texttt{-d pln}
tell SUAMC to download the PLN (Plant) GenBank division. The
\texttt{-i Onagraceae} and \texttt{-o Lythraceae}
tells SUMAC to search the PLN division for all sequences within
the taxonomic groups Onagraceae (the ingroup) and Lythraceae (the outgroup).
SUMAC will then perform all-by-all BLAST comparisons of each sequence, 
build clusters of putatively homologous sequences, and 
construct a supermatrix. 

Unless you are on a large multi-core system, the all-by-all BLAST comparisons
will take a very long time to be performed since well over 5000 sequences
will be found.
To speed up the supermatrix construction, you could make a FASTA file
of guide sequences to define each cluster. Each guide
sequence could be an example of a sequence commonly used for phylogenetic
analysis. 
You could then use this command:

\begin{verbatim}
python -m sumac -d pln -i Onagraceae -o Lythraceae -g guides.fasta
\end{verbatim}

Which approach is better for constructing supermatrices?
Using guide sequences makes supermatrix construction much faster, however
it requires a priori knowledge of which DNA regions will be used
in the supermatrix.
Performing all-by-all BLAST comparisons is computationally
more expensive, but it effectively data-mines GenBank in an exploratory
fashion, so that sequence data not necessarily used in previous systematic
studies can also be incorporated into the supermatrix.
The decision will depend on the size of the taxonomic group being
analyzed and the computational resources available.

\section{Explanation of the Output Files}

SUMAC will output the following files:

\begin{enumerate}
\item \texttt{alignments/combined.fasta} The final aligned supermatrix in FASTA format.
\item \texttt{alignments/N.fasta} The alignment of gene region $N$ where $N$ is an integer $> 0$.
\item \texttt{clusters/N.fasta} The unaligned raw sequence cluster of gene region $N$.
\item \texttt{gb\_search\_results} File used by SUMAC to save the results of the GenBank sequence search in case the search is re-run. This file is not human readable.
\item \texttt{genbank\_accessions.csv} A table with each GenBank accession used, ordered by gene region and taxon (like the appendices found in most systematics papers).
\item \texttt{gene\_regions.csv} A table with the number of taxa, the aligned length, the percent missing data, and the taxon coverage density of each gene region used in the supermatrix.
\item \texttt{plot.pdf} A figure that shows how much sequence data was available for each taxon for each gene region.
\item \texttt{sumac\_log} Log of the SUMAC run, and contains a great deal of information about the supermatrix construction, including final metrics such as the partical decisiveness (PD) of the supermatrix.
\end{enumerate}

% start comment
\iffalse
The \verb|alignments/combined.fasta| is the final aligned supermatrix in FASTA format.
\verb|alignments/N.fasta|, where $N$ is an integer $> 0$, is the alignment of gene region $N$.
Similarly, \verb|clusters/N.fasta| is the unaligned raw sequence cluster of gene region $N$.
The \verb|gb_search_results| file is used by SUMAC to save the results of the GenBank
sequence search in case the search is re-run. This file is not human readable.

The two CSV (comma-separated values) files contain tables that provide useful
summary information about the supermatrix. \verb|genbank_accessions.csv| is a table
with each GenBank accession used, ordered by gene region and taxon (like the
appendices found in most systematics papers). The \verb|gene_regions.csv|
file contains the number of taxa, the aligned length, the percent missing data,
and the taxon coverage density of each gene region used in the supermatrix.

The \verb|plot.pdf| file is a figure that shows how much sequence data was available for
each taxon for each gene region.
\verb|sumac_log| is a log of the SUMAC run, and contains a great deal of information
about the supermatrix construction, including final metrics such as the partical
decisiveness (PD) of the supermatrix.
\fi
% end comment

%----------------------------------------
% SUMAC in Detail
%----------------------------------------

\chapter{SUMAC in Detail}

\section{Downloading GenBank}

\subsection{GenBank Division}

The first time you run SUMAC you must specify which GenBank division
to download with the \verb|-d div| option, where \verb|div| is the
GenBank designated three letter code of the division (PLN, MAM, etc).
Once SUMAC has downloaded the GenBank division, future SUMAC runs
may leave out the \verb|-d div| option to avoid repeatedly
download the same files.

\subsection{GenBank File Path}

By default, each SUMAC run searches for the downloaded GenBank files
in \verb|./genbank/|, a subdirectory of the current run's directory.
It may be useful to save the GenBank files outside of the current working
directory, in which case 
you can specifiy the absolute path of the GenBank files with the \verb|-p path| option.
For example, if you want to build
multiple supermatrices (or different versions of the same one) each in a different working directory
it is helpful
to use \verb|-p /genbank| so that all SUMAC runs use the same copy of the GenBank files.

\section{Specifying Ingroup and Outgroup}

The \verb|-i| and \verb|-o| options must be used to specify which ingroup and
outgroup to search for. The taxonomic names must be those used by GenBank.
If a SUMAC run is repeated with the same ingroup and outgroup, SUMAC
will load the previous search results to save time.

\section{Using Guide Sequences}

Guide sequences should be in a single standard FASTA file specified
using the \verb|-g| option. The names of the
guide sequences will be ignored, and each of the ingroup and outgroup 
sequences will be BLASTed against the guide sequences.

\section{Homologous Sequence Thresholds}

\subsection{BLAST E-value}
By default, SUMAC uses a threshold default BLASTn e-value $1.0e-10$.
This can be changed with the \verb|-e| option.

\subsection{Sequence Length Similarity}
SUMAC uses a default threshold of sequence length percent similarity
of 0.5. This can be changed with the \verb|-l| option.

\section{Partial Decisiveness}
blah blah

\section{Supermatrix Figure}
blah blah




\end{document}
