\documentclass[12pt]{report}
\usepackage{geometry} 
\usepackage{indentfirst}
\usepackage{hyperref}
\usepackage{color}
\usepackage{comment}
\usepackage[pdftex]{graphicx}  
\usepackage{caption}
\usepackage{natbib}
\usepackage{mathtools}
\usepackage{units}
\usepackage{booktabs}
\usepackage{authblk}
\renewcommand{\baselinestretch}{1.5}
\geometry{a4paper} 
\bibliographystyle{apalike}

%----------------------------------------
%AUTHORS
%----------------------------------------
\title{SUMAC: Supermatrix Constructor version 1.0 Manual}
\author{William A. Freyman\thanks{freyman@berkeley.edu}}
\affil{Department of Integrative Biology, University of California, Berkeley}

\date{}

%-----------------------------------------------------------------------------------------------------------------
% BEGIN DOCUMENT
%-----------------------------------------------------------------------------------------------------------------
\begin{document}
\maketitle

\tableofcontents

%----------------------------------------
% INTRODUCTION
%----------------------------------------

\chapter{Introduction}

\section{About SUMAC}

blah blah

\section{Installation}

blah blah

%----------------------------------------
% Quick Start Tutorial
%----------------------------------------

\chapter{Quick Start Tutorial}

\section{Construct a Simple Supermatrix}

blah blah

\section{Explanation of the Output Files}

blah blah


%----------------------------------------
% SUMAC in Detail
%----------------------------------------

\chapter{SUMAC in Detail}

\section{Downloading GenBank}

\subsection{GenBank Division}
blah blah
\subsection{GenBank File Path}
blah

\section{Specifying Ingroup and Outgroup}

blah b

\section{Using Guide Sequences}
blah blah

\section{Partial Decisiveness}
blah blah

\section{Supermatrix Figure}
blah blah




\end{document}
